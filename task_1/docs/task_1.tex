\documentclass[a4paper,10pt,ngerman]{scrartcl}
\usepackage{babel}
\usepackage[T1]{fontenc}
\usepackage[utf8x]{inputenc}
\usepackage[a4paper,margin=2.5cm,footskip=0.5cm]{geometry}

% Die nächsten vier Felder bitte anpassen:
\newcommand{\Aufgabe}{Aufgabe 1: Schiebeparkplatz} % Aufgabennummer und Aufgabennamen angeben
\newcommand{\TeamId}{00067}                       % Team-ID aus dem PMS angeben
\newcommand{\TeamName}{Panic! at the Kernel}                 % Team-Namen angeben
\newcommand{\Namen}{Christopher Besch}           % Namen der Bearbeiter/-innen dieser Aufgabe angeben
 
% Kopf- und Fußzeilen
\usepackage{scrlayer-scrpage, lastpage}
\setkomafont{pageheadfoot}{\large\textrm}
\lohead{\Aufgabe}
\rohead{Team-ID: \TeamId}
\cfoot*{\thepage{}/\pageref{LastPage}}

% Position des Titels
\usepackage{titling}
\setlength{\droptitle}{-1.0cm}

% Für mathematische Befehle und Symbole
\usepackage{amsmath}
\usepackage{amssymb}

% Für Bilder
\usepackage{graphicx}

% Für Algorithmen
\usepackage{algpseudocode}

% Für Quelltext
\usepackage{listings}
\usepackage{color}
\definecolor{mygreen}{rgb}{0,0.6,0}
\definecolor{mygray}{rgb}{0.5,0.5,0.5}
\definecolor{mymauve}{rgb}{0.58,0,0.82}
\lstset{
  keywordstyle=\color{blue},commentstyle=\color{mygreen},
  stringstyle=\color{mymauve},rulecolor=\color{black},
  basicstyle=\footnotesize\ttfamily,numberstyle=\tiny\color{mygray},
  captionpos=b, % sets the caption-position to bottom
  keepspaces=true, % keeps spaces in text
  numbers=left, numbersep=5pt, showspaces=false,showstringspaces=true,
  showtabs=false, stepnumber=2, tabsize=2, title=\lstname
}
\lstdefinelanguage{JavaScript}{ % JavaScript ist als einzige Sprache noch nicht vordefiniert
  keywords={break, case, catch, continue, debugger, default, delete, do, else, finally, for, function, if, in, instanceof, new, return, switch, this, throw, try, typeof, var, void, while, with},
  morecomment=[l]{//},
  morecomment=[s]{/*}{*/},
  morestring=[b]',
  morestring=[b]",
  sensitive=true
}

% Diese beiden Pakete müssen zuletzt geladen werden
\usepackage{hyperref} % Anklickbare Links im Dokument
\usepackage{cleveref}
\usepackage[shortlabels]{enumitem}

% Daten für die Titelseite
\title{\textbf{\Huge\Aufgabe}}
\author{\LARGE Team-ID: \LARGE \TeamId \\\\
	    \LARGE Team-Name: \LARGE \TeamName \\\\
	    \LARGE Bearbeiter/-innen dieser Aufgabe: \\ 
	    \LARGE \Namen\\\\}
\date{\LARGE\today}

\begin{document}

\maketitle
\tableofcontents

\vspace{0.5cm}

\section{Lösungsidee}
% Die Idee der Lösung sollte hieraus vollkommen ersichtlich werden, ohne dass auf die eigentliche Implementierung Bezug genommen wird.
Im Folgenden werden querstehende Autos mit \glqq{}Q-Auto\grqq{} abgekuerzt; alle nicht querstehenden Autos werden \glqq{}Auto\grqq{} genannt.
Ein Feld---eine moegliche Position fuer ein Q-Auto---selbst kann nicht verschoben werden, allerdings wird im Folgenden diese Terminologie verwendet.
Hierbei wird festgelegt, dass ein Feld soweit verschoben werden kann, wie das an dieser Stelle stehendes Q-Auto verschoben werden kann.
Wenn kein Q-Auto auf einem Feld steht, kann es beliebig verschoben werden.

Es laesst sich feststellen, dass jedes Feld in einer von drei Kategorien eingeordnet werden kann:
\begin{enumerate}
    \item Das Feld kann zweifach nach links verschoben werden.
          Eine Verschiebung um mehr als zwei Positionen ist nie notwendig da die Laenge eines Q-Autos zwei entspricht.
          Diese Verschiebungen werden ignoriert.
    \item Das Feld kann genau einfach nach links verschoben werden.
    \item Das Feld kann nicht nach links verschoben werden.
\end{enumerate}
Diese Kategorien gelten analog fuer Rechtsverschiebungen.

Nun kann fuer jedes Feld $A$ bestimmt werden, welche anderen Felder wieweit verschoben werden muessen, damit $A$ seiner Kategorie entsprechend maximal verschoben werden kann.
Hierbei werden alle Felder, die nicht mit einem Q-Auto besetzt sind per Definition als nicht zu verschieben angesehen.
Diese Information wird linke beziehungsweise rechte volle Verschiebungsbestimmung des Feldes $A$ genannt.
Eine Verschiebungsbestimmung $\Lambda$ fuer das Feld $A$ bei insgesamt $n$ Feldern entspricht der Liste mit den Indizes $0,1,\dots,n-1$.
Das Element $e$ am Index $i$ entspricht $1$, wenn es einfach verschoben werden muss, $2$ wenn es zweifach verschoben werden muss und $0$ wenn es nicht verschoben werden muss.

Wenn ein Feld zweifach verschoben werden kann, das Feld allerdings nur einfach verschoben werden muss, kann die reduzierte Verschiebungsbestimmung verwendet werden kann.
Die reduzierte Verschiebungsbestimmung entspricht der vollen Verschiebungsbestimmung $\Lambda$, wobei jeder Wert $e$ von $\Lambda$ um $1$ reduziert wird, wenn $e>0$.
Dies laesst sich dadurch zeigen, dass wenn ein Q-Auto einmal weniger verschoben werden muss alle anderen notwendigen Verschiebungen ebenfalls einmal weniger durchgefuehrt werden muessen.
Die fuer eine bestimmte Verschiebung notwendige Verschiebungsbestimmung (entweder voll oder reduziert) wird notwendige Verschiebungsbestimmung genannt.

\medskip
Nun muss fuer jedes Feld sowohl die linke und rechte Kategorie als auch die Verschiebungsbestimmungen vorliegen.
Damit laesst sich recht einfach bestimmen:
\begin{itemize}
    \item Ob ein beliebiges Auto ausfahren kann:
          Dies ist der Fall wenn mindestens eine Kategorie des betroffenen Feldes mindestens Kategorie 1 entspricht.
    \item Ob die Verschiebung nach links oder recht optimal ist:
          Wenn die Verschiebung nach links weniger Q-Autos bewegt als die nach rechts ist sie optimal.
          Die Menge an zu verschiebenden Autos entspricht der Haelfte der Menge an Elemente $e$ in der notwendigen Verschiebungsbestimmung mit $e>0$.
    \item Welche Q-Autos wie verschoben werden muessen, damit das betreffende Auto ausfahren kann.
          Hierzu kann wie bereits beschrieben die notwendige Verschiebungsbestimmung angewendet werden.
          Es ist nur zu beachten, dass sowohl das linke als auch rechte Feld des Q-Autos in der Verschiebungsbestimmung vorkommen.
          Diese Doppelung muss ausgefiltert werden.
\end{itemize}

\subsection{Bestimmung der Kategorien und Verschiebungsbestimmungen}
Das Verfahren wird fuer die Linksverschiebungen beschrieben, kann allerdings ebenfalls fuer Rechtsverschiebungen angewendet werden.

Wenn an einem Feld kein Q-Auto vorhanden ist, kann es zweifach verschoben werden.
Andernfalls gilt:
\begin{itemize}
    \item Damit ein Feld einfach verschoben werden kann muss das direkt links am Q-Auto angrenzende Feld mindestens einfach verschiebbar sein.
    \item Damit ein Feld zweifach verschoben werden kann muss das links am Q-Auto angrenzende Feld zweifach verschiebbar sein.
          Zudem muss das uebernaechste linke Feld mindestens einfach verschiebbar sein.
\end{itemize}
Diese Information kann rekursiv berechnet werden, wobei der Rekursionsanker dem linken Rand entspricht.
Der Rand kann nicht verschoben werden.

Fuer die Verschiebungsbestimmungen kann ein aehnliches Verfahren angewendet werden.
Die Summe $\Alpha$ zweier Verschiebungsbestimmungen $\Lambda$ und $\Gamma$ entspricht fuer jedes Element aus $\Alpha$ dem Maximum der jeweiligen Element aus $\Lambda$ und $\Gamma$.
\begin{itemize}
    \item Wenn ein Feld nur einfach verschoben werden kann entspricht dessen Verschiebungsbestimmung der notwendigen Verschiebungsbestimmung fuer die einfach Verschiebung des links angrenzenden Feldes.
    \item Kann es zweifach verschoben werden entspricht sie der Summe aus
          \begin{enumerate}[a)]
              \item der notwendigen Verschiebungsbestimmung fuer die zweifache Verschiebung des direkt angrenzenden Feldes und
              \item der notwendigen verschiebungsbestimmung fuer die einfach Verschiebung des uebernaechsten Feldes.
          \end{enumerate}
\end{itemize}
Diese Information kann ebenfalls rekursiv bestimm werden.
Die Rekursionsanker sind die freien Felder

Es wird das Prinzip der dynamischen Programmierung verwendet.

\section{Umsetzung}
% Hier wird kurz erläutert, wie die Lösungsidee im Programm tatsächlich umgesetzt wurde. Hier können auch Implementierungsdetails erwähnt werden.

\section{Beispiele}
% Genügend Beispiele einbinden! Die Beispiele von der BwInf-Webseite sollten hier diskutiert werden, aber auch eigene Beispiele sind sehr gut – besonders wenn sie Spezialfälle abdecken. Aber bitte nicht 30 Seiten Programmausgabe hier einfügen!

\section{Quellcode}
% Unwichtige Teile des Programms sollen hier nicht abgedruckt werden. Dieser Teil sollte nicht mehr als 2–3 Seiten umfassen, maximal 10.

\end{document}
