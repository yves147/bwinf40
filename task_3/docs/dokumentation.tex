\documentclass[a4paper,10pt,ngerman]{scrartcl}
\usepackage{babel}
\usepackage[T1]{fontenc}
\usepackage[utf8x]{inputenc}
\usepackage[a4paper,margin=2.5cm,footskip=0.5cm]{geometry}

% Die nächsten vier Felder bitte anpassen:
\newcommand{\Aufgabe}{Aufgabe 3: Wortsuche} % Aufgabennummer und Aufgabennamen angeben
\newcommand{\TeamId}{00067}                       % Team-ID aus dem PMS angeben
\newcommand{\TeamName}{Panic! at the Kernel}                 % Team-Namen angeben
\newcommand{\Namen}{Yves Léon Jansen}           % Namen der Bearbeiter/-innen dieser Aufgabe angeben
 
% Kopf- und Fußzeilen
\usepackage{scrlayer-scrpage, lastpage}
\setkomafont{pageheadfoot}{\large\textrm}
\lohead{\Aufgabe}
\rohead{Team-ID: \TeamId}
\cfoot*{\thepage{}/\pageref{LastPage}}

% Position des Titels
\usepackage{titling}
\setlength{\droptitle}{-1.0cm}

% Für mathematische Befehle und Symbole
\usepackage{amsmath}
\usepackage{amssymb}

% Für Bilder
\usepackage{graphicx}

% Für Quelltext
\usepackage{listings}
\usepackage{color}
\definecolor{mygreen}{rgb}{0,0.6,0}
\definecolor{mygray}{rgb}{0.5,0.5,0.5}
\definecolor{mymauve}{rgb}{0.58,0,0.82}
\lstset{
  keywordstyle=\color{blue},commentstyle=\color{mygreen},
  stringstyle=\color{mymauve},rulecolor=\color{black},
  basicstyle=\footnotesize\ttfamily,numberstyle=\tiny\color{mygray},
  captionpos=b, % sets the caption-position to bottom
  keepspaces=true, % keeps spaces in text
  numbers=left, numbersep=5pt, showspaces=false,showstringspaces=true,
  showtabs=false, stepnumber=2, tabsize=2, title=\lstname
}
\lstdefinelanguage{JavaScript}{ % JavaScript ist als einzige Sprache noch nicht vordefiniert
  keywords={break, case, catch, continue, debugger, default, delete, do, else, finally, for, function, if, in, instanceof, new, return, switch, this, throw, try, typeof, var, void, while, with},
  morecomment=[l]{//},
  morecomment=[s]{/*}{*/},
  morestring=[b]',
  morestring=[b]",
  sensitive=true
}

% Diese beiden Pakete müssen zuletzt geladen werden
\usepackage{hyperref} % Anklickbare Links im Dokument
\usepackage{cleveref}
\usepackage[shortlabels]{enumitem}

% Daten für die Titelseite
\title{\textbf{\Huge\Aufgabe}}
\author{\LARGE Team-ID: \LARGE \TeamId \\\\
	    \LARGE Team-Name: \LARGE \TeamName \\\\
	    \LARGE Bearbeiter/-innen dieser Aufgabe: \\ 
	    \LARGE \Namen\\\\}
\date{\LARGE\today}

\begin{document}

\maketitle
\tableofcontents

\vspace{0.5cm}

\section{Lösungsidee}
% Die Idee der Lösung sollte hieraus vollkommen ersichtlich werden, ohne dass auf die eigentliche Implementierung Bezug genommen wird.
\subsection{Schwierigkeiten}
Das Programm kann Lösungen in drei Schwierigkeiten generieren.
Für den Menschen ist es am einfachsten, in Richtung links-nach-rechts und oben-nach-unten zu lesen.
Daher generiert die erste Stufe lediglich Wörter in diese Richtungen für das Rätsel.
Schwieriger wird es dann, wenn auch schräg-geschriebene Wörter möglich sind. Zudem werden in der
zweiten Schwierigkeitsstufe Teile der Wörter in das Rätsel eingefügt, um für Verwirrung zu sorgen.
Als letzte Schwierigkeit können Wörter nun in alle Richtungen geschrieben sein. Dazu werden
die Buchstaben auch zufällig groß und klein geschrieben. Das macht das Rätsel nochmal unübersichtlicher. 
\subsection{Schritte}
Das Hauptproblem bei dieser Aufgabe ist es, die Wörter so nebenbeinander und übereinander anzuordnen, dass
Sie alle in die verfügbaren Zellen des Rätsels passen. Der Ansatz geht nach folgenden Schritten vor:
\begin{enumerate}
	\item Sortiere die Wörter nach Länge, sodass die längsten Wörter am Anfang sind.
	\item Für jedes Wort: Suche einen zufälligen Punkt $(x, y)$ aus, der noch nicht von anderen Buchstaben b
	esetzt ist.
	\item Schaue, ob von diesen ausgehend in eine zufällige Richtung das Wort eingesetzt werden kann. Alle 
	Felder müssen entweder frei sein oder denselben Buchstaben beinhalten, der dort eingesetzt werden muss.
	\item Falls ja, setze es ein. Sonst, versuche es erneut ab Schritt 2.
\end{enumerate}
Die oberen Punkte können jedoch nicht immer zu einer Lösung führen, da gewisse Kombinationen an schon
gesetzten Wörtern keine Möglichkeiten mehr bieten, die Restlichen korrekt einzusetzen.
Falls keine Lösung nach mehreren Durchläufen gefunden werden kann, wird also ein neuer Versuch gestartet.

Zudem ist noch ein Spezialfall zu berücksichtigen, der durch die obere Vorgehensweise nicht ausreichend 
abgedeckt wird. Wenn man ein Wort hat, dass genauso lang ist, wie das Feld breit bzw. hoch ist, wird bei der 
Generierung des Startpunktes $(x,y)$ berücksichtigt, dass er nur am Rand existieren kann und das Wort nur in 
gewisse Richtungen sicher verlaufen kann, ohne die anderen Worte zu beeinflussen.

\section{Umsetzung und Quellcode}
Die Lösungsidee wurde in C++ umgesetzt. Da C++ keine plattformübergreifende Implementation von Umlauten hat, 
kann das Programm nicht mit diesen umgehen. Die Richtungen sind als Enum im Programm dargestellt. Ein 
einzelner Punkt in den Feld nxm, dessen Größe durch die Eingabe vorgegeben wird, ist durch die Structure point 
beschrieben. Das Attribut $i$ steht hierbei für die Y-Koordinate und $k$ für die X-Koordinate.

\begin{lstlisting}[language=C++]
enum richtung {
    HOCH,
    RUNTER,
    RECHTS,
    LINKS,
    RUNTER_LINKS,
    RUNTER_RECHTS,
    HOCH_LINKS,
    HOCH_RECHTS
};

struct point {
    int i, k;
};
\end{lstlisting}

Das Feld, in das die Wörter später eingefügt werden, wird als 2D-Vektor vom Typ \lstinline{wchar_t} 
beschrieben. Nachdem die Wörter als string-Array eingelesen worden wird im Fall das eine Lösung mit der 
höchsten Schwierigkeiten generiert werden soll, eine Methode namens \lstinline{extraWoerter} aufgerufen, 
welche Wörterteile als neue Wörter in das Wörter-Array hinzufügt. Andernfalls wird dies übersprungen.

\begin{lstlisting}[language=C++]
void extraWoerter(int groesse)
{
    for(int i = 0; i < 10; ++i) {
        int     rindex           = rand() % groesse;
        wstring wordOriginal     = woerter[rindex];
        int teilGroesse          = std::min(5 + rand() % 3, (int)wordOriginal.size() - 1);
        wstring wordDanach       = wordOriginal.substr(0, );
        woerter[groesse - i - 1] = wordDanach;
    }
}
\end{lstlisting}

Als Nächstes wird für jedes Wort die Funktion \lstinline{wortEinsetzen} aufgerufen, welche versucht, das Wort 
in die verbleibenden Felder einzubetten. Hier wird wie bereits beschrieben erstmal ein zufälliger Punkt 
ausgesucht und eine zufällige Richtung bestimmt, jedoch abhängig von der Schwierigkeit, 
was hier als switch-case implementiert wurde.
Die möglichen Punkte werden als \lstinline{unordered_set} von Koordinaten dargestellt (Im folgenden Code-
Ausschnitt die Variable \lstinline{verfuegbar}). 

\begin{lstlisting}[language=C++]
auto rt     = rand() % verfuegbar.size();
     n      = *select_random(verfuegbar, r);
switch(SCHWIERIGKEIT) {
case 0:
    rt = richtung(rand() % 3);
    break;
case 1:
    rt = richtung(rand() % 6);
    break;
case 2:
    rt = richtung(rand() % 8);
}
\end{lstlisting}

Im Fall, dass ein Wort das ganze Feld ausfüllt, begrenzt das Programm die Möglichkeiten dieses Wortes, in das Feld eingesetzt zu werden.

\begin{lstlisting}[language=C++]
pair<int, int> n;
if(laenge == feld[0].size()) {
	// Fuegt alle Punkte, die ganz links im Feld sind, in einen
	// eigenen Vektor hinzu
	vector<pair<int, int>> t;
	for(int hi = 0; hi < feld[0].size(); ++hi) {
	pair<int, int> j   = std::make_pair<int, int>((int)hi, 0);
	auto           sit = verfuegbar.find(j);
	if(sit != verfuegbar.end()) {
	t.push_back(*sit);
	}
	}
	// zufaellger Punkt aus denen, die ganz links im Feld sind
	auto r = rand() % t.size();
	n      = *select_random(t, r);
	rt      = richtung::RECHTS;
}
else if (laenge > feld[0].size() && laenge > feld.size()) {
	// Fuegt alle Punkte, die ganz oben im Feld sind, in einen
	// eigenen Vektor hinzu
	vector<pair<int, int>> t;
	for(int hi = 0; hi < feld.size(); ++hi) {
	pair<int, int> j   = std::make_pair<int, int>(0, (int)hi);
	auto           sit = verfuegbar.find(j);
	if(sit != verfuegbar.end()) {
	t.push_back(*sit);
	}
	}
	// zufaelliger Punkt aus denen, die ganz oben im Feld sind
	auto r = rand() % t.size();
	n      = *select_random(t, r);
	rt      = richtung::RUNTER;
}
\end{lstlisting}

Nachdem also ein zufälliger verfügbarer Punkt ausgesucht wurde, ist zu überprüfen, ob dort das jeweilige Wort 
eingesetzt werden kann. Das passiert, indem jeweils für das Wort abhängig von einer Richtung und einen 
Startpunkt, eine Methode \lstinline{kannEinsetzen} aufgerufen wird.

\begin{lstlisting}[language=C++]
bool kannEinsetzen(vector<vector<wchar_t>>& feld, const wchar_t* wort, point start, richtung d)
{
    int   i = 0, l = (int)std::char_traits<wchar_t>::length(wort);
    point np = start;
    while(i < l) {
    	// np.i ist hier die Y-Koordinate und np.k die X-Koordinate
    	// Wenn diese ausserhalb des erlaubten Bereiches sind,
    	// kann das Wort nicht eingesetzt werden
        if(np.i < 0 || np.i >= feld.size() || np.k < 0 || np.k >= feld[0].size()) {
            return false;
        }
        // unbesetzt oder derselbe Buchstabe, der eingesetzt werden wuerde
        if(feld[np.i][np.k] == STANDARD || feld[np.i][np.k] == wort[i]) {
            punktBewegen(feld, d, np);
            i++;
        }
        else {
            return false;
        }
    }
    return true;
}
\end{lstlisting}
Wenn das Wort am Ende passt, werden alle Buchstaben an den passenden Koordinaten in den 2D-Vektor eingesetzt.
Sobald ein Feld besetzt wird, wird auch die Koordinate aus den möglichen Punkten entfernt.
\begin{lstlisting}[language=C++]
int   i  = 0;
point np = start;
while(i < laenge) {
	// entferne Punkt aus den Moeglichen
    verfuegbar.erase(make_pair(np.i, np.k));
    if(SCHWIERIGKEIT == 2) {
    	// Bei hoeherer Schwierigkeit wird zu 50% auch ein kleingeschriebener Buchstabe
    	// eingesetzt
        if(rand() % 2 == 0) {
            feld[np.i][np.k] = (wchar_t)towupper(wort[i]);
        }
        else {
            feld[np.i][np.k] = (wchar_t)towlower(wort[i]);
        }
     }
     else {
         feld[np.i][np.k] = (wchar_t)towupper(wort[i]);
     }
     punktBewegen(feld, rt, np);
     i++;
}
\end{lstlisting}
%Wie oben beschrieben, wird, für den Fall, dass eine gewisse Kombination an besetzten Feldern keine Lösung mehr bietet - also das Einsetzen eines speziellen Wortes bei vielen Versuchen nicht ermöglicht, der vorherige Fortschritt (damit sind die schon eingesetzten Worte gemeint) gelöscht und somit aus dem 2D-Array entfernt.

Zu dem Zeitpunkt, an dem das Programm eine passende Lösung zur Zusammensetzung gefunden hat, sind also auch
schon die zuvor durch \lstinline{extraWoerter} generierten Wortteile in der Zwischenlösung enthalten.
Jetzt füllt die Methode \lstinline{fuellen} die restlichen Felder mit zufälligen Buchstaben. 
Auf höchster Schwierigkeit werden hier auch die Buchstaben zufällig groß- oder kleingeschrieben befüllt.
Hierfür wird für jedes Feld, dass noch dem Standard-Charakter entspricht, die Methode \lstinline{zufallsCharakter} aufgerufen.

\begin{lstlisting}[language=C++]
wchar_t zufallsCharakter()
{
    if(SCHWIERIGKEIT == 2) {
        if(rand() % 2 == 0) {
            return 'A' + rand() % 26;
        }
        else {
            return 'a' + rand() % 26;
        }
    } else {
        return 'A' + rand() % 26;
    }
}
\end{lstlisting}

\section{Beispiele}
Das Programm braucht für alle Beispieldateien ungefähr 3-6 Millisekunden.
\subsection*{Aufgabenbeispiel auf höchster Schwierigkeit}
\begin{lstlisting}
cat examples/worte0.txt | ./a.out 2

    V   D
    o A  
    r    
  T o r f
a v e    

Z K V X D
k w o A v
L T r v g
B T o r f
a v e U T
\end{lstlisting}
\subsection*{worte3.txt im einfachen Modus}
\begin{lstlisting}
cat examples/worte3.txt | ./a.out 0
## WOERTER ALLEINE
                                               
                                               
            I                                  
            N   E M I S S I O N                
            F             E M P A T H I E      
            E                                  
            K                                  
      K     T       N                          
      O     I       O D                        
      N     O       I E             N          
      J     N       T K           L O          
      U             I O           E I          
      N             U R           G T       R  
      K         V   T A         C I U       E  
      T         E   N T         H T L       F  
      U         R   I I         R I O       E  
      R         S     O         O M V       R  
                      N         N A E       A  
                                I T R       T  
                                K I            
            M O N O G R A M M     O            
                                  N            

## ERGEBNIS
C V B E M C R C O N X V N Z O O I L X I D V D Z
V Q Z R R H Y V E Z C R D T T U J S P Y S E M A
R J K W G P I V D F U U Y E U U I V W Z Z R V V
A N M A F Q N M E M I S S I O N H J X U F G J D
J R X G P B F A J K W H L E M P A T H I E X Z G
S B U E C B E X Q J G Q D O W O R F F Q N X U O
G G K O R J K P Y B Q U F S V E J G N B L B X A
U E H K K S T E E G N N M S D E B S E E K Y L D
V Q O O C D I P N G O D O P A S Z M K D T S S F
M X J N N P O P R A I E N F F L V V N N A K C G
A S I J S R N U F W T K N X R T K L O O E Z G U
T G J U Y N U V I H I O V U M D U E I E L O X F
T V V N T M O F B U U R E H O X P G T Y X E R U
S O E K P I Q U V E T A W F U T C I U N O O E K
J B Y T O E U U E N N T J U E J H T L R I D F M
W H D U G Z Y M R V I I F G O U R I O W F E E F
G E V R M Z Q Z S I K O D T B N O M V Y P J R I
T R H R D D Y J T T G N Y X N G N A E C J S A B
Z T L L Y G O L E E W M Z N W Q I T R Q Z Q T A
U J I V H V B J G W M H R Z S Q K I F H B J N A
W M N U E E M O N O G R A M M V U O G P E O K L
L O W S M K Z D J S V O B Z Z O Z N V A M R G S
\end{lstlisting}
\subsection*{worte4.txt in allen Schwierigkeiten}
\begin{lstlisting}
cat examples/worte0.txt | ./a.out 0
## WOERTER ALLEINE
    E           C O M M O N S P         X L A B R U F D A T U M
  E D             K           A         O I           E   K    
  T N             A   E     E M         B T           T   A    
  A E             T   L     R E     B   O E O         I   L    
  N G             E L L     L T N   E   X R S         E   E    
  I E S           G A E   U E A O   L   A A T         S   N    
  D L T G         O N U   E D N I       T T E         S   D    
Z R B R N   B     R G Q   D I I T         U R E       N   E    
N O R A U   E     I   O T F G D A Y     T R R C       O   R S  
F O A H T   L   O E   E E R T R M R     H   E R     D I   S I  
  C F C H   E   F G   G X A   O R A   B C   I U     N S E T E  
      K C   G   N R     T     O O N E D E   C O     A S M I W  
    O I A   E   I A           C F O L M R   H S     L U A L N  
    R S   I P A   P             N I L I N   B I     O K R   I  
    P U           H             I T E P E   E K   E O S F   H B
      M A B S A T Z S O R T       K T I P   Z I   T B I V   S E
M E D A I L L E N S P I E G E L   I S N P   O W   S S D A   G G
                                  W U G A   G     I N   N   N R
                                    A   W   E   P E O       U I
    M                               B       N   O L M W   C R F
Z   L F O L G E N L E I S T E       B G R N     S N M I   O A F
I   I         I N T E R N E T Q U E L L E A     I E O K   M L S
T   F                                 E   V     T N C I N M K K
A                                     B   I     I O O D A O S L
T L I Z E N Z U M S T E L L U N G   G A   G B   O S N A C N F A
I       D D B                       E B N A E   N R   T   S F R
O   A R C H I V B O T               R   E T N   S E   A   C I U
N                     B I B R E C O R D T I U   K P   W   A R N
              F U S S B A L L D A T E N S O T   A     E C T G G
                                        A N Z V R     B H   E  
                      T                 K S E I T     A A   B  
    B B K L   E       U   N               L R H E     R R      
              H       A   F               E   C       C T      
V             O               C O L       I   R       H S      
I             H   A R C H I V I E R U N G S   A       I        
H M U L T I L I N G U A L       A         T   O       V        
C         D O I                 U         E   T                
R       S M I L E Y             S             U                
A                             C E N T E R     A                
A L L M U S I C                                         G N I S

## ERGEBNIS
B R E H Y E X E C O M M O N S P I O X K X L A B R U F D A T U M
P E D C N N G T B K W Q V R U A I R M Y O I H P J S S E C K B Q
I T N Y U Q M T D A B E X Q E M Q E J S B T C A P T W T Y A M I
Y A E T Z J N R L T Q L H V R E O D B M O E O C W R D I T L H U
X N G Q O C S F X E L L O D L T N M E W X R S D F V O E V E O Z
L I E S X V C A J G A E E U E A O W L C A A T K S J E S P N B U
T D L T G U Z R I O N U F E D N I D H I T T E K E X F S T D Y T
Z R B R N Q B V V R G Q R D I I T G B P I U R E L H T N R E Z V
N O R A U L E S P I L O T F G D A Y L Z T R R C S O I O C R S B
F O A H T N L B O E I E E R T R M R I A H B E R Y X D I Y S I R
G C F C H C E J F G Q G X A N O R A Q B C K I U F R N S E T E F
Q M W K C D G X N R X X T O G O O N E D E A C O R T A S M I W B
Z B O I A D E D I A B D D V L C F O L M R F H S E B L U A L N T
W N R S Y I P A O P S Q C Q W Z N I L I N O B I T P O K R U I V
G P P U X G S C L H T F O O S U I T E P E U E K U E O S F N H B
R H N M A B S A T Z S O R T H C F K T I P J Z I S T B I V B S E
M E D A I L L E N S P I E G E L K I S N P I O W X S S D A C G G
E S I T R Q L W B H D S V W M S Q W U G A B G L Z I N O N S N R
B V D W Z P G W R L Q C E K U S G Y A Z W L E Q P E O W H C U I
O Z M F Z Z V T C Q Y Y P O F N H Q B G L W N Q O L M W I C R F
Z O L F O L G E N L E I S T E Y G P B G R N J W S N M I M O A F
I R I A A S F I N T E R N E T Q U E L L E A C R I E O K C M L S
T X F U S W U K M Z X V R F H O V P E E V V X U T N C I N M K K
A F V I Y V K Q C M K E K E Y I Y K U B A I K R I O O D A O S L
T L I Z E N Z U M S T E L L U N G R G A P G B Y O S N A C N F A
I H L P D D B O I O K P J S N E C G E B N A E I N R P T S S F R
O O A R C H I V B O T B Y P L Y Z F R Y E T N J S E Y A S C I U
N Z X B M O P U D Z K B I B R E C O R D T I U M K P T W Z A R N
T W H D N Z U F U S S B A L L D A T E N S O T Q A Z J E C T G G
B Y I I W U G Q T F S H W J B Z L N N E A N Z V R P I B H A E Y
O P A I F Z R G X B O T U V X M O C E X K S E I T Y O A A Z B Z
Z O B B K L P E F F Z U G N E N X E V D D L R H E M L R R D O A
Z L Z L A B Q H Z C G A Y F D H P S M U U E U C A H S C T G D V
V K R G P H I O O T J Z V N R C O L U P Z I V R U G K H S P V G
I J P J Q J P H V A R C H I V I E R U N G S G A B N P I Q W Z G
H M U L T I L I N G U A L P J I A M W Z H T N O B C H V J O W E
C W F W F D O I Y H X T O Y I F U P F E V E W T N D J L F Q Y I
R U I R S M I L E Y K E Y I M W S Q K F N D V U L O A C T F A R
A U F A I K S I U M Q M Y X M C E N T E R H J A K X V P K A L V
A L L M U S I C O L A J S A B P F B X S U H M I Y A H V G N I S

\end{lstlisting}
\begin{lstlisting}
cat examples/worte0.txt | ./a.out 1
## WOERTER ALLEINE
              B B K L                           W              
                      E T R A K S N O I T I S O P I            
          W                     B E L               K          
A           I                                         I        
B             K   Z T A X O B O X       F       G E R   D      
R           F   T   I     I N F O R M A T I O N         O A    
U         B R   P I B T   C                 L           R   T  
F     M     A     E O I A   O                 M         P     A
D   S E       B     R N B T   M     M U L T I L I N G U A L    
A   T D         E     S A R I   M     N E T A D L L A B S S U F
T B R A           L     O R E O   O             R           C G
U E A I                   N Y C N   N   N F     G       E   O N
M N H L                     E   O     S         B   C N R   O U
  U C L         B E L E G E   N   R     C           E E L   R L
  T K E   V I H C R A O T U A   L   D     A         N T E   D L
T Z I N         S N O M M O C O N E         T       T S D   I E
X E S S C   S M I L E Y             I           G   E A I   N T
E R U P H         A R C H I V         S       I   N R K G   A S
T   M I A           S N O M M O C       T       M   I   T   T M
A     E R       C O L       A U S         E       D   S     E U
  U   G T       D I S K U S S I O N S S E I T E     B       M Z
    T E S                   E D N E G E L B R A F           A N
E     L                                   A                 P E
M         W A P P E N R E C H T           B   O F N I         Z
A           E L L E T S U A B S O R T     S                   I
R                                         A           G E     L
F       N E G O Z E B H C I E R R E T S O T   I O D N N C      
V     B O O L A N D               U E D   Z         A A R      
A             E T S I E L S N O I T A G I V A N     C L U      
N           E L L E U Q T E N R E T N I         I   W G O      
E T A N I D R O O C     K A T E G O R I E G R A P H E E S     L
  K A L E N D E R S T I L                       A   B O I     I
      B E G R I F F S K L A R U N G S H I N W E I S A Q K     T
                                                    R U I     E
        E T S I E L N E G L O F         G           C E W     R
    E H O H           G N U T H C A     N     Z     H L       A
      A R C H I V I E R U N G           I     N     I L       T
G N U R A L K S F F I R G E B           P     F     V E       U
          A L L M U S I C           A R C H I V B O T         R
      B D D                                                    

## ERGEBNIS
O N B B P N N B B K L M S M L F U X I P N D N L W O Y G Q W K W
L A E D O R E R G R F E T R A K S N O I T I S O P I U L T I T N
F B C V G W R I X R R N O B L Z B E L D Q C U J I L K S A S M N
A L U G A C I J E X S V F P P W C D Z P C F F H B P P I O J R H
B V F U S N V K U Z T A X O B O X X B T F R Z A G E R G D O Z I
R R A Z V H F F T E I I W I N F O R M A T I O N U X F N O A G C
U S A W I B R X P I B T S C H B L Y A O E O L P O H P N R C T X
F U H M H Q A D G E O I A X O Q M A L P Y T O M R A P E P Y P A
D S S E E F H B S M R N B T Z M F R M U L T I L I N G U A L Y C
A Y T D F S D M E S T S A R I A M U E N E T A D L L A B S S U F
T B R A P S Z G J L Z V O R E O O O Z N I F U S R X H R F A C G
U E A I R H Y Y B D M T W N Y C N N N N N F D E G I C K E T O N
M N H L E I H F X S L R K K E A O D P S A W Y A B U C N R A O U
D U C L H V C W B E L E G E L N F R A T C J L M P V E E L W R L
X T K E S V I H C R A O T U A Q L K D L C A U M H M N T E O D L
T Z I N D K P X S N O M M O C O N E M U G K T W C V T S D D I E
X E S S C D S M I L E Y Q N Q E E N I A B F S N G Q E A I U N T
E R U P H I E B U A R C H I V V E G K S E V E I M N R K G F A S
T D M I A P C G U S S N O M M O C V M W T A Z Z M D I G T T T M
A R Y E R O B D C O L R V Y A U S Y D K E E Y Q Q D D S T F E U
F U C G T B Z X D I S K U S S I O N S S E I T E N Y B X M Z M Z
D S T E S S U R G V W A S W E D N E G E L B R A F Y V I C W A N
E A S L Z U A H Y B I V R G V F F Y A A U A T I P R I J P G P E
M G X I C W A P P E N R E C H T X D E T F B L O F N I R I V N Z
A Z C I H J E L L E T S U A B S O R T I H S J C C S T T C C I I
R J K H T N E Y R A D E R M Z G L D Q U N A Y B Y C F G E S V L
F Y V Z N E G O Z E B H C I E R R E T S O T I I O D N N C G K P
V B Z B O O L A N D U A Q U D W N U E D R Z X W F B A A R M Z Q
A N D O P I J E T S I E L S N O I T A G I V A N N J C L U E M R
N M Y H N X E L L E U Q T E N R E T N I D P P Z I V W G O L O M
E T A N I D R O O C L K K A T E G O R I E G R A P H E E S R O L
Z K A L E N D E R S T I L T E M W U D G D Q P J A F B O I J V I
D S L B E G R I F F S K L A R U N G S H I N W E I S A Q K B V T
A S W V G M I R X B F W U L K S F N Z L E Q U J Z S R U I O R E
D R O F E T S I E L N E G L O F K M B Q G Z L J W O C E W O U R
J Z E H O H E D G R C G N U T H C A R V N V O Z G U H L G V M A
L M C A R C H I V I E R U N G Q Y P T O I O G N X K I L V N E T
G N U R A L K S F F I R G E B G O K J V P B O F O Z V E J C H U
F K C T V A L L M U S I C R W O P L A R C H I V B O T J G Z Q R
F L N B D D T R T H B E E E U T D F V L L I P E F I C T A N F G

\end{lstlisting}

In der folgenden Ausgabe kann man auch erkennen, wie z.B. "CHArTs" nach oben-links gerichtet ist, was für den Menschen nochmal eindeutig schwerer zu lesen ist.	
\begin{lstlisting}
cat examples/worte0.txt | ./a.out 2
## WOERTER ALLEINE
                                c E N T e R                   m
  o f n i m u l T I L i n G U A L E     e L L e u q O e g   M r
        W I K I s O u r C E     G           S m I l E y     u o
      C i S u m l L A b     H E       a b R u F D a t U M   S f
          G E R     a     O L                   a         U I n
    p B   F       b   g H e         P e R s O N E U     e   k I
  R   E   F     e   N E B                           s D     C t
O     G   i   L   i         l I t E R a T U R               h A
      r   R F U S s b A L l d A T E N   c o m m O N s c a t A r
      I   G                                 V i H C R a     r e
      F   E   E t R a k s n O I t I s O p           P       t t
  e   F   b                             e     b     I       s i
  t   S P A M E T A n i d r O O C       L N   d n   N       n l
n S   K   B o O l a N D                 l A   m e   G       E  
o i   l         p e R s o N e           e V   I g       c   t  
i E   A g n U L l e t s m u Z N E Z i L U I     o E   L A   S  
t L   R   a t A D I K i W               q G     Z R   e n   a t
a n   u     t   L t E X t       s o R t T a     e l   g     K a
t e   n   u   K                         e T     b e   E       D
I n   G a   B b E n u t z E R   d       n I     h d   i       I
Z o   s   B   D i S k u s         d     r o     c I   p       k
  s   H                         M   b   e n   E i g   s       I
  R V I h c r a b E W g n a L     l     T s   M e T   N   y   w
  e   n   G N U T H C A     e g B   I   n L   a R     e   r   b
  p   w T h c E R n E p p A w t N A   F i e   r R     L   a   G
      e         L a N e   l   N a u u     I   f e     L   N   r
    Z I     i       T     E     O n r s   S   v T     i   O s  
    t s     p     s       b B     C i a t T   a S     A   i N e
    A       A   i   a     N   I     O d L E   N o     D   t o L
    S     i   E   r         f   B     m R K           E   K m L
L   b e o   l   c                 R   s m o S         m   i m E
  O a D g n   H       x O b o X a t e   T o o f           w o t
    C   E l i                         C   r N c F         F c s
      G   v O   L I T s r E d n e l A K O   A S   I     N     U
    L   b     f f A r b l e g E n D E     r   H     R z       A
  O   O       A                             D   c     G       b
F   t       R     A R c H I V i e R u N G               E      
          f       D I S K U s s I o n s S E i t e         B    
              n O I t a M R O f N i                            
  V i h c r a O t u a               K A T e G O r i e g R a P h
  
  
## ERGEBNIS
f B M h Y s R C Q y L y d w a K c E N T e R M c Y j y r h P B m
i o f n i m u l T I L i n G U A L E Q S e L L e u q O e g G M r
L O q G W I K I s O u r C E B D G F N x T w S m I l E y E J u o
l t t C i S u m l L A b B Z H E I m n a b R u F D a t U M V S f
y S h k j G E R i M a v o O L l z L t T S D Z l a K L R o U I n
q M p B B F w J l b H g H e O v P R P e R s O N E U e D e y k I
L R Z E y F i i e D N E B n z t w F O t B K q p a C s D P O C t
O h I G B i t L w i n A F e l I t E R a T U R e O n J x d u h A
P Q H r S R F U S s b A L l d A T E N n c o m m O N s c a t A r
o o s I K G P T O u c z K A J o n k u W T M V i H C R a I O r e
J K i F l E B E t R a k s n O I t I s O p w U p q K P c Z s t t
H e S F Z b g q W k o m U s F X N h l i e p i b u O I D P j s i
z t j S P A M E T A n i d r O O C s W a L N q d n Q N b I H n l
n S N K u B o O l a N D t e o q V j q i l A S m e c G e j v E j
o i w l O B D K p e R s o N e k v y f H e V j I g K P F c n t Y
i E s A g n U L l e t s m u Z N E Z i L U I D J o E Q L A e S E
t L z R O a t A D I K i W M e J M k o h q G L C Z R Y e n x a t
a n I u u C t q L t E X t A p B s o R t T a u a e l i g I c K a
t e r n M u s K h M p X k m d a e G S c e T M b b e N E j X y D
I n P G a u B b E n u t z E R s d B w u n I I p h d U i x M J I
Z o x s W B S D i S k u s H r j l d I p r o M h c I a p u o Z k
k s R H Q e r c K O Z w d O l A M v b w e n x E i g P s z n U I
u R V I h c r a b E W g n a L H Y l m S T s c M e T u N T y N w
X e X n C G N U T H C A H j e g B U I o n L J a R x e e e r c b
I p W w T h c E R n E p p A w t N A s F i e U r R a j L G a D G
r x o e E q y K L a N e E l W N a u u x R I d f e l V L g N K r
a g Z I Q b i U e Z T Y n E l P O n r s V S E v T V R i m O s n
H d t s L j p f Z s h d T b B K w C i a t T H a S a W A T i N e
h O A M m q A h i Z a F I N W I P I O d L E n N o b z D v t o L
Z x S N l i e E v r l H D q f Z B v M m R K y w y i l E g K m L
L b b e o U l Q c b d F A V V x y R q s m o S Q T U b m o i m E
v O a D g n C H F C C x O b o X a t e t T o o f d I J V f w o t
z P C a E l i S N W M y o Y b x d R J C L r N c F X g t P F c s
o z m G u v O B L I T s r E d n e l A K O J A S V I N O N Y d U
i M L B b Q N f f A r b l e g E n D E d K r n H U Y R z u m v A
N O c O G W c A b P Z E m r L j m o a g W q D K c S o G d E b b
F C t y k u R h I A R c H I V i e R u N G n n e e Z D f E D p d
y i L H O f D j L D I S K U s s I o n s S E i t e x f P L B O i
y w N I K S K n O I t a M R O f N i b n I F k X r q q M O w b R
g V i h c r a O t u a s y w T x y D K A T e G O r i e g R a P h
\end{lstlisting}
\subsection*{worte5.txt im einfachen Modus}

\begin{lstlisting}
## WORT ALLEINE
                                                           
                                                           
                                                           
                                                           
                                                           
                                                           
                                                           
                                                           
                                                           
                                                           
                                                           
                                                           
                                                           
                                                           
                                                           
                                                    D      
                                                    A      
                                                    S      
                                                           
                                                           
                                                           
                                                           
                                                           
                                                           
                                                           
                                                           
                                                           
                                                           
                                                           
                                                           
## ERGEBNIS ("das" ist zur Erkennbarkeit nachher kleingemacht worden)
B V E P E G O Z K F R V H S N F L U H I W F Z H R X Y T U C
P V Y V N E E B D O H W J Q P W Y C S H L O M M V D J U Y D
W P A W L N B P R E D A B O S S M Q U E Z H V L V Q Q E M R
J L G M H T Z I K S P P S S D L K R D G Y D Q T Q L J H S W
Y B J G N Q C P Z M H Q C C I H N S Z S A X V Q S M E B V W
Z V Z I B N B F E C S L S U P A D C U E X U D S N V G R Z B
P Y Y O J C D M H H O Z V G V K I Z P C D M Y J G L E N E D
Q V E R K N T P B A Z P C U X Z G F Y V J E J H P Q V T F B
X X X B O J Q H Y R K X I O T F N A M O V V U F E J X B C C
D B B C C Q L U Z L N J L V X E C L E O B C L V J Q E I R G
M W I N Y M D J H D X W O I S M M U Z T L A X W V I M B Q G
H C C R P B G V M P Y J L M T D A G A B B L D A H Y I W B A
C L C E C T H K O U Z O F L B Z O D H O F K Z I K J J U F M
U H X Y N C R V M I R M W Y X Z Z N F G C M S D W C O H Y T
U S C R S Q T K N G U G S S E R S E E X M I L F O H J C P I
W J C Y C V Q W H D E B L Y T S P N Y T M K E X R S d F B U
W L Q F N R H K H F R N J U Y H Q Q Y F O U S B Y R a S Q Y
V K U G D Z W U J G D O A S A W T H O L I U A C O D s A H X
Q G T C A B F C Z B L I F Z K Z Z G S J X E T T G V J J X Q
I P W B R Z F X B E A M O F N Y F P H Z A E D T X J O G U N
Y D C V G T U L S X S T J G A Y F H P O H P S K K R W A X Q
N X T P S A J O N B L H W W O X X V E M J L C D Y O U U P T
M C Q I S L K D B X E N H D L V A K S G X D S B I S P C O E
X A H N I B A S E C S K R Z N C W N N Q U M V O N D I E H W
J E Y S T J T U D X W X H P W V T U K I M G U H U J M E O V
A Z Z B R V K M R P J P N S E L P X I A G U I C E F M S J C
O M B N N U K Z G B O R Q D J U P Z U X B A T J E X O Q Q Y
U E K V T Z R G A X H Q Q Y U A U L B O K C Q D L V D C N T
C I Z O F U N Z C P W K F N K B P G M Q V W U N C H I F J W
A L G B Z N X O M A D L M L A W M P C B F Z Z B N B I X I S
\end{lstlisting}

\subsection*{worte2.txt - Spezialfall 1}
\begin{lstlisting}
## WOERTER ALLEINE
R u T a t S a t    
    C O M P u t    
      u s B        
r e T U P M o C    
    u p M O C      
                   
B i l D S C H i R m
F e S t p l A T t E
                   
        m a U s    

## ERGEBNIS
R u T a t S a t A F
k H C O M P u t g f
S c j u s B a R K z
r e T U P M o C n d
s r u p M O C l C R
P s I R g x p v h O
B i l D S C H i R m
F e S t p l A T t E
X U g r X S J Y g n
K M U t m a U s r p
\end{lstlisting}
Hier sieht man, dass BILDSCHIRM und FESTPLATTE richtig in das Rätsel eingefügt werden.
Zudem sind in der zweiten Zeile zur Verwirrung "COMPut" und "upMOC" ("COMPU") als Teile von COMPUTER
eingefügt wurden.\medskip
\subsection*{Weiterer Spezialfall}
Spezielle Datei:
\begin{lstlisting}
5 5
5
ABSTAND
\end{lstlisting}
In diesen Beispiel ist es nicht möglich das Wort einzusetzen. Es wird der Fehler "Keine Loesung" ausgegeben.

\end{document}
