\documentclass[a4paper,10pt,ngerman]{scrartcl}
\usepackage{babel}
\usepackage[T1]{fontenc}
\usepackage[utf8x]{inputenc}
\usepackage[a4paper,margin=2.5cm,footskip=0.5cm]{geometry}

% Die nächsten vier Felder bitte anpassen:
\newcommand{\Aufgabe}{Aufgabe 1: Schiebeparkplatz} % Aufgabennummer und Aufgabennamen angeben
\newcommand{\TeamId}{00067}                       % Team-ID aus dem PMS angeben
\newcommand{\TeamName}{Panic! at the Kernel}                 % Team-Namen angeben
\newcommand{\Namen}{Christopher Besch}           % Namen der Bearbeiter/-innen dieser Aufgabe angeben
 
% Kopf- und Fußzeilen
\usepackage{scrlayer-scrpage, lastpage}
\setkomafont{pageheadfoot}{\large\textrm}
\lohead{\Aufgabe}
\rohead{Team-ID: \TeamId}
\cfoot*{\thepage{}/\pageref{LastPage}}

% Position des Titels
\usepackage{titling}
\setlength{\droptitle}{-1.0cm}

% Für mathematische Befehle und Symbole
\usepackage{amsmath}
\usepackage{amssymb}

% Für Bilder
\usepackage{graphicx}

% Für Algorithmen
\usepackage{algpseudocode}

% Für Quelltext
\usepackage{listings}
\usepackage{color}
\definecolor{mygreen}{rgb}{0,0.6,0}
\definecolor{mygray}{rgb}{0.5,0.5,0.5}
\definecolor{mymauve}{rgb}{0.58,0,0.82}
\lstset{
  keywordstyle=\color{blue},commentstyle=\color{mygreen},
  stringstyle=\color{mymauve},rulecolor=\color{black},
  basicstyle=\footnotesize\ttfamily,numberstyle=\tiny\color{mygray},
  captionpos=b, % sets the caption-position to bottom
  keepspaces=true, % keeps spaces in text
  numbers=left, numbersep=5pt, showspaces=false,showstringspaces=true,
  showtabs=false, stepnumber=2, tabsize=2, title=\lstname
}
\lstdefinelanguage{JavaScript}{ % JavaScript ist als einzige Sprache noch nicht vordefiniert
  keywords={break, case, catch, continue, debugger, default, delete, do, else, finally, for, function, if, in, instanceof, new, return, switch, this, throw, try, typeof, var, void, while, with},
  morecomment=[l]{//},
  morecomment=[s]{/*}{*/},
  morestring=[b]',
  morestring=[b]",
  sensitive=true
}

% Diese beiden Pakete müssen zuletzt geladen werden
\usepackage{hyperref} % Anklickbare Links im Dokument
\usepackage{cleveref}
\usepackage[shortlabels]{enumitem}

% Daten für die Titelseite
\title{\textbf{\Huge\Aufgabe}}
\author{\LARGE Team-ID: \LARGE \TeamId \\\\
	    \LARGE Team-Name: \LARGE \TeamName \\\\
	    \LARGE Bearbeiter/-innen dieser Aufgabe: \\ 
	    \LARGE \Namen\\\\}
\date{\LARGE\today}

\begin{document}

\maketitle
\tableofcontents

\vspace{0.5cm}

\section{Lösungsidee}
% Die Idee der Lösung sollte hieraus vollkommen ersichtlich werden, ohne dass auf die eigentliche Implementierung Bezug genommen wird.
Fuer jedes Hotel laesst sich sagen, ob es innerhalb von $d$ Tagen erreicht werden kann.
Ist dies fuer ein $d$ der Fall, kann fuer die minimale Bewertung aller auf dem optimalen Weg zu ihm liegenden Hotels angegeben werden.
Dieser optimale Weg benoetigt genau $d$ Tage.
Zudem ist er in dem Sinne optimal, dass kein anderer Weg existiert, der in $d$ Tagen das Hotel erreicht und eine besserer minimale Bewertung aufweist.
Diese Werte koennen in einer Tabelle dargestellt werden:
\begin{lstlisting}
            min rating at day:
idx rating  0   1   2   3   4   5
=== ======  =   =   =   =   =   =
0   inf     inf
1   4.3         4.3
2   4.8         4.8 4.3
3   2.7         2.7 2.7 2.7
4   2.6         2.6 2.6 2.6 2.6
5   3.6             3.6 3.6 2.7 2.6
6   0.8             0.8 0.8 0.8 0.8
7   4.4             2.7 3.6 3.6 2.7
8   2.8                 2.7 2.8 2.8
9   2.6                 2.6 2.6 2.6
10  2.1                     2.1 2.1
11  2.8                     2.7 2.8
12  3.3                         2.7
13  inf                         2.7
\end{lstlisting}
Neben den Hotels werde hier mit dem Index $0$ der Startort und mit $13$ das Ziel gelistet.
Dessen Bewertung ist nie schlechter als die eines beliebigen Hotels, weshalb sie als unendlich angesehen wird.
Da eine Fahrt, die laenger als fuenf Tage dauert nicht zulaessig ist, werden diese Optionen weggelassen.

Anhand dieser Daten laesst sich recht leicht die minimale Bewertung auf dem Weg zu jedem beliebigen Hotel ausgeben.
Interessant ist dieser Wert fuer das Ziel.
Es muss lediglich das Minimum aller Werte in der letzten Zeile bestimmt werden.

Um nun die einzelnen Hotels, die optimal zum Ziel fuehren, zu bestimmen, wird eine zweite Tabelle benoetigt.
Diese weist die gleichen Dimensionen auf, gibt allerdings fuer jedes Hotel $x$ und Tag $d$ das Hotel, das am vorherigen Tag zuletzt besucht wurde, an.
So kann vom Ziel am optimalen Tag ausgehend immer das zuletzt besuchte Hotel bestimmt werden.
Dieser Vorgang endet, wenn man am Anfang angekommen ist.

\subsection{Berechnung der Tabellen}
Als Taglaenge wird die Strecke verstanden, die an einem Tag zurueckgelegt werden kann.
Zu jedem Hotel $x$ kann man am Tag $d$ von allen Hotels, die maximal eine Taglaenge vor $a$ liegt und innerhalb von genau $d-1$ Tagen erreichbar sind, gelangen.
Diese Hotels bilden die Menge $Y$.
Alle Hotels, die hinter $x$ liegen, sind nicht zu verwenden.
Die minimale Bewertung zum Hotel $x$ nach $d$ Tagen betraegt mit der minimalen Bewertungen $z_y$ zum Hotel $y$ nach $d-1$ Tagen mit $y\in Y$:
\begin{equation*}
    min(z_1,  z_2, \dots , z_n)
\end{equation*}

Da so immer nur die Information von Hotels, die vor dem aktuellen liegen, benoetigt werden, kann das Prinzip der dynamischen Programmierung verwendet werden.
Hierbei werden die Tabellen anfangend beim Start Hotel fuer Hotel gefuellt.
Jedes weitere Hotel benoetigt ausschlisslich die bereits berechnete Information.

\section{Umsetzung}
% Hier wird kurz erläutert, wie die Lösungsidee im Programm tatsächlich umgesetzt wurde. Hier können auch Implementierungsdetails erwähnt werden.

\section{Beispiele}
% Genügend Beispiele einbinden! Die Beispiele von der BwInf-Webseite sollten hier diskutiert werden, aber auch eigene Beispiele sind sehr gut – besonders wenn sie Spezialfälle abdecken. Aber bitte nicht 30 Seiten Programmausgabe hier einfügen!

\section{Quellcode}
% Unwichtige Teile des Programms sollen hier nicht abgedruckt werden. Dieser Teil sollte nicht mehr als 2–3 Seiten umfassen, maximal 10.

\end{document}
